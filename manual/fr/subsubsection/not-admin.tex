\documentclass[manual-fr.tex]{subfiles}
\begin{document}
Il faut alors ajouter le dossier où se trouve l'exécutable wapiti à la
variable PATH. Il est préférable de d'abord créer une copie de cette variable
via la commande "PATH\_OLD=\$PATH". Il faut ensuite ajouter le dossier à la fin
de la variable PATH via la commande :\\

PATH=\$PATH:\~{}/foobar/stand-alone-tagger/ext/wapiti-X\\

S'il y a eu une erreur dans la commande de rajout du dossier à la variable PATH, il suffit de
restaurer sa valeur de base via la commande "PATH=\$PATH\_OLD".\\

/!\textbackslash Les modifications faites à la variables PATH ne sont valables que dans le
terminal où elles sont écrites et ne perdurent pas après la fermeture de ce
dernier. Il faut donc refaire cette étape à chaque nouveau terminal ouvert.
\end{document}