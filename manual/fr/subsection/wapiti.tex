\documentclass[manual-fr.tex]{subfiles}
\begin{document}
\WapitiCite\ est un logiciel implémentant les CRF, il permet d'apprendre des annotateurs à partir de corpus annotés.\\

La dernière version de \Wapiti\ compatible avec \SEM\ est disponible dans le
dossier ext. Les consignes d'intallation sont disponibles avec. \SEM\ est prévu
pour fonctionner avec le \Wapiti tel qu'il est fourni dans le dossier ext,
il faut le compiler pour pouvoir appeler Wapiti avec \SEM.\\

Supposons que stand-alone-tagger se trouve dans le dossier foobar dans le
dossier utilisateur. Sous Linux, son chemin absolu est donc \$HOME/foobar
ou \~{}/foobar. stand-alone-tagger se trouve donc à l'emplacement
\~{}/foobar/stand-alone-tagger.\\

Dans le dossier \~{}/foobar/stand-alone-tagger/ext se trouve une archive de la
forme wapiti-X.tar.gz où X est un numéro de version. Il faut extraire cette
archive dans le dossier même. Cela crééra un dossier nommé wapiti-X.\\

dans le dossier \~{}/foobar/stand-alone-tagger/ext/wapiti-X, tapez la commande
"make" pour créer l'exécutable wapiti. Si vous êtes sous Windows, utilisez
la commande ".\\make.bat" (le \Wapiti disponible dans ext peut être compilé
avec MinGW).

\subsubsection{Erreurs de compilation}
\subfile{fr/subsubsection/compilation-error}
\end{document}
