\documentclass[manual-fr.tex]{subfiles}
\begin{document}

\begin{itemize}
    \item[] \textbf{description}
        \begin{itemize}
            \item[] Il s'agit du module principal de \SEM. Il permet d'effectuer une chaîne de traitements sur un fichier.
                Ces traitements correspondent à des modules ou à des annotations faites à l'aide de \Wapiti. Les modules à
                enchaîner et l'ordre dans lequel cet echaînement s'effectue est donné par un fichier de configuration xml
                appelé fichier de configuration maître.
        \end{itemize}
    \item[] \textbf{arguments}
        \begin{itemize}
            \item[] master : fichier xml
                \begin{itemize}
                    \item[] le fichier de configuration maître. Définit le séquencage des traitements à effectuer.
                \end{itemize}
            \item[] input\_file : fichier
                \begin{itemize}
                    \item[] le fichier d'entrée. Peut être soit un fichier de texte brut soit un fichier vectorisé.
                \end{itemize}
        \end{itemize}
    \item[] \textbf{options}
        \begin{itemize}
            \item[] --help ou -h : switch
                \begin{itemize}
                    \item[] affiche l'aide
                \end{itemize}
            \item[] --output-directory ou -o : dossier
                \begin{itemize}
                    \item[] le répertoire où les fichiers temporaires vont être créés (défaut : dossier courant).
                \end{itemize}
        \end{itemize}
\end{itemize}

\end{document}