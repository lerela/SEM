\documentclass[manual-fr.tex]{subfiles}
\begin{document}
Il faut alors aller dans un terminal et taper la commande suivante :\\

bzr branch lp:~yoann-dupont/crftagger/stand-alone-tagger\\

Cela va créer un dossier stand-alone-tagger dans le répertoire où est tapée la commande. \\

Il s'agit de la branche bazaar (dépôt), qui sert à gérer les différentes
versions du logiciel. Il ne faut en AUCUN cas modifier le contenu de ce
dossier (c'est surtout vrai si on prévoit de mettre-à-jour la branche, mais
c'est une habitude à prendre immédiatement). Pour utiliser \SEM, il faut copier
les différents fichiers et dossiers dans un autre répertoire. Un dossier .bzr
est présent : étant caché il ne sera pas copié si on n'active pas l'affichage
des fichiers cachés, sinon il faut le déselectionner. C'est ce dossier qui
contient les informations de versionnement.\\

L'intérêt ici est de pouvoir mettre-à-jour simplement le logiciel en tapant la
commande "bzr up" dans la branche. Cela mettra à jour uniquement les fichiers
qui doivent l'être, ce qui est pratique quand (comme ici) le contenu est assez
lourd.
\end{document}